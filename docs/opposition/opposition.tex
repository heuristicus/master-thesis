\documentclass[11pt,a4paper]{article}

\usepackage{fontspec}
\defaultfontfeatures{Mapping=tex-text}
\setromanfont[Ligatures={Common},Numbers={Lining}]{Linux Libertine}
\author{Michal Staniaszek}
\title{Opposition Report: \emph{Visual Odometry for Autonomous MAV with On-Board
  Processing} by Jacob Greenberg}

\begin{document}
\maketitle
In his project, Jacob deals with the problem of using visual odometry to track
the position of an MAV. The reason for using visual odometry is that there are
some environments in which the standard method of GPS cannot be used, such as
inside buildings. Specifically, he evaluates the efficacy of a particular
algorithm, AICK, for this purpose. The aims of the project are well explained,
both in the abstract and the introduction sections. The title is a good
reflection of these aims, but perhaps including that the work is for GPS-denied
environments would be good. Justification for the method of tackling the problem
is also given in the introduction, and while it is not particularly detailed,
given that the project aims to investigate the method, the explanation given is
sufficient. Additional reasons for investigating AICK are given in the related
work section.

The ideas behind the project are adequately explained, with the background
section giving information about how MAVs work, and hints at why visual odometry
may be needed, as well as some information about the RGB-D camera, registration
and visual odometry. This covers all of the major parts of the project. The
related work section explains in more detail the ideas behind visual odometry
and specific approaches to the problem. Some approaches to RGB-D SLAM are also
mentioned, which is relevant as position tracking forms a major part of SLAM
algorithms.

Chapter 4 describes in detail the ORB and AICK methods, explaining how features
are extracted from point cloud frames and then matched between frames. Each part
is explained with all the important details, and one can understand the logic
behind the method well from reading the chapter. It also discusses some
different approaches for the registration in terms of which frames are used,
which forms an important component of the experimental results.

The implementation chapter seems a little out of place in its current position
in the report. While hardware details and things relating to the implementation
should be mentioned in the report, the flow is broken slightly by this chapter,
and I feel it would be better as an appendix.

The experimental evaluation is well done. Several trials are performed with
characteristics for the motion of the MAV to investigate registration quality.
These include two trials where the performance of the system is not
satisfactory. The inclusion of these trials is good, as it acknowledges the
existence of environments in which the system does not perform well, and
examines the problems in more depth. Each trial has a detailed written
evaluation along with figures showing registration results and information about
the keypoints extracted. Although the purpose of these trials was to test the
registration, it would have been good to see the tracking error for the
trajectories, but this is not something that is easy to gather, so its omission
is not a serious problem given the scope of the project.

Trials are also performed to compare the tracking error on the ORB-SLAM and AICK
methods. It is acknowledged that the experimental setup for these trials was not
ideal and might affect the results to some degree. However, the trials are still
useful to see the performance of the system, and are the only realistic way to
evaluate the tracking error. The plots shown give a good idea of how the
tracking drifts for the two methods.

One potential issue with the layout of the experimental evaluation is that some
techniques for solving problems with the system are proposed there, rather than
in the method section. While they are quite small, since they are part of how
the final system works perhaps it would be better to move the explanation of
those modifications. At the same time, the flow of the experiments section might
be broken by the removal of the explanations.

The final section of the chapter mentions autonomous station-keeping
experiments, but data from these experiments is not shown --- it would be nice
to see this as well.

Since there was only one type of keypoint selection method and descriptor used,
it might have been instructive to see how different selection methods or
descriptors would affect the system.

To summarise, the experimental evaluation acknowledges the limitations of the
system, and proposes solutions to certain problems encountered during testing,
some of which are implemented. This is good as it indicates that solutions to
these problems were considered.

The conclusion briefly explains what was achieved in the project, and mentions
some of the issues that were encountered. The future work section mentions some
specific improvements that could be made to improve the system, and also
suggests some other sensor setups that could be used to navigate GPS-denied
environments.

The bibliography contains papers which are very relevant to the work. Some
landmark papers on the SIFT and SURF features are included, as well as some
papers describing corner detection methods. Papers for the ORB and AICK methods
are also present. ICP and visual odometry literature are also included. Some
papers on benchmarking and evaluating SLAM are mentioned, as well as some work
specifically on MAVs with constrained computational power. All of these
inclusions mean that there are no glaring omissions in the bibiliography.

There was one section of the report which was difficult to understand. On page
45 there is a discussion of how monocular SLAM is unable to compute the scales,
which I could not grasp very well. Other than that, beyond minor spelling and
grammatical mistakes there are no issues, particularly in the sections which are
important for understanding. Page 42, with the explanation of parts of the scene
with large numbers of keypoints exiting the image is also not very easy to follow.

Overall, I think the project is good, and has achieved quite good results. A
number of different scenarios were examined, showing the strengths and
weaknesses of the approach. Some solutions to problems encountered were
proposed, and some of them made a big difference to the end result. All but a
few small parts of the report were easy to understand.

\section*{Questions}
  \begin{itemize}
  \item Advantages/disadvantages of visual odometry over e.g. laser sensors
  \item Different keypoint algorithm could help with the problem of low
    keypoint numbers?
  \item Why is rotation more difficult to handle (6.2.2)
  \item How could you test the station keeping of (6.4)
  \item Could visual odometry also be useful outside of GPS-denied environments?
  \end{itemize}
\end{document}